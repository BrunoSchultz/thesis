% !TEX root = mythesis.tex

%==============================================================================
\chapter{Data Reduction}
\label{sec:data_reduction}
%==============================================================================
In the following section, the underlying data shall be reduced and corrected for the various effects and contamination sources explained in Section \ref{sec:background}. Data from eRASSX is utilized for for all TMs (1-7) using \textit{eROSITA} pipeline processing version c010. The galaxy group NGC1550 is located in skytile 065087. In addition, the surrounding skytiles 062084, 062087, 062090, 065084, 065090, 068084, 068087, 068090 are used to encompass regions up to \(\sim 3R_{200}\). The data reduction is performed with the software HEASoft version XXX and the extended Science Analysis Software System (eSASS 4DR1). Images were created using astropy.
%
\section{Raw photon images}
Before the data reduction process, raw photon images for all combined skytiles and TMs are presented across the following energy bands: \SIrange{0.2}{2.3}{\kilo\electronvolt}, \SIrange{2.3}{6.0}{\kilo\electronvolt}, and \SIrange{6.0}{9.0}{\kilo\electronvolt}. The skytiles were combined using the \textit{eSASS} task \texttt{evtool}, with no additional parameters applied to reveal all inherent deficiencies in the raw images. The raw photon images are presented in \ref{fig:raw_photon_images}. As observed, most cluster emission is concentrated in the lower energy band, making it the focus for detecting emission structures. In this analysis, the \SIrange{0.2}{2.3}{\kilo\electronvolt} energy band is used. Due to the light-leak in TM9, however, the energy range is restrictated to \SIrange{0.8}{2.3}{\kilo\electronvolt}, while for TM8 it remains \SIrange{0.2}{2.3}{\kilo\electronvolt}. Hereafter, this TM-dependent energy band will be referred to as the \enquote{soft band} and the \SIrange{6.0}{9.0}{\kilo\electronvolt} energy range as the \enquote{hard band}. 
\begin{figure}[htbp]
    \centering
    \begin{subfigure}{0.32\textwidth}
        \centering
        \includegraphics[width=\textwidth,height=\textwidth,keepaspectratio]{data_reduction/combined_tiles_0_raw_0.2-2.3keV.pdf}
        \caption{\SIrange{0.2}{2.3}{\kilo\electronvolt}}
        \label{fig:low_energy}
    \end{subfigure}
    \hfill
    \begin{subfigure}{0.32\textwidth}
        \centering
        \includegraphics[width=\textwidth,height=\textwidth,keepaspectratio]{data_reduction/combined_tiles_0_raw_2.3-6.0keV.pdf}
        \caption{\SIrange{2.3}{6.0}{\kilo\electronvolt}}
        \label{fig:mid_energy}
    \end{subfigure}
    \hfill
    \begin{subfigure}{0.32\textwidth}
        \centering
        \includegraphics[width=\textwidth,height=\textwidth,keepaspectratio]{data_reduction/combined_tiles_0_raw_6.0-9.0keV.pdf}
        \caption{\SIrange{6.0}{9.0}{\kilo\electronvolt}}
        \label{fig:high_energy}
    \end{subfigure}
    \caption{Raw photon images from TM0 of all combined skytiles centered around NGC1550, displayed in the energy bands \SIrange{0.2}{2.3}{\kilo\electronvolt}, \SIrange{2.3}{6.0}{\kilo\electronvolt}, and \SIrange{6.0}{9.0}{\kilo\electronvolt}, with Gaussian smoothing of 4 pixels applied. The colorbar represents (smoothed) photon counts. Most of the cluster emission is visible in the lower energy band (\SIrange{0.2}{2.3}{\kilo\electronvolt}). A noticeable count drop on the right side of the images is evident and will be addressed through the exposure map correction detailed in Section X.}
    \label{fig:raw_photon_images}
\end{figure}
%
\section{Image filtering}
Each skytile is cleaned individually using \texttt{evtool} with \texttt{pattern=15} to select all event patterns and \texttt{flag=0xc00fff30} to remove bad pixels and CCD corners. Subsequently, soft proton flares are identified and mitigated through the following process: the \texttt{flaregti} task is used to generate light curves with \SI{20}{\second} time bins in the energy range of \SIrange{5}{10}{\kilo\electronvolt}. A \(3\sigma\) threshold is determined; time intervals exceeding this threshold indicate elevated count rates likely due to soft proton flares. The task \texttt{flaregti} is then rerun using this threshold to establish good-time-intervals (GTIs) excluding these flare periods, which are applied using \texttt{evtool} with the \texttt{gti="FLAREGTI"} parameter. All SPF-filtered and cleaned skytiles are combined into a single TM0 photon image using \texttt{evtool}. Hereafter, these combined images shall be referred to as \enquote{filtered}. The \texttt{evtool} task with the \texttt{telid} parameter is also used to split the filtered photon images into individual filtered images for each TM, as needed for subsequent steps.
%
\section{PIB-Correction}
Subtracting the particle-induced background (PIB) from the combined filtered photon image is necessary. The following  approach is based on extensive studies of the \textit{eROSITA} FWC data conducted by Dr. F. Pacaud, as utilized for example in \cite{Reiprich2021}.  The PIB is modeled for each TM using filter wheel closed (FWC) data. Due to the minimal spatial variation of the PIB, this modeling utilizes the flat exposure map created with the \textit{eSASS} task \texttt{expmap}. Furthermore, given the negligible spectral variation, the counts \(H_\text{obs}\) in the hard band, where PIB counts dominate, is used to estimate the PIB contribution in the soft bands by multiplication with the ratio \(R\) of the number of FWC counts in the soft band \(S_{\text{FWC}}\) to the hard band \(H_\text{FWC}\). A background map for each TM is then generated by applying this factor to the flat exposure map, normalized to 1 by dividing by the sum of all pixel values (norm. exposure map). Hence, the PIB map of a given TM is given by
\begin{align*}
    \text{PIB map}_\text{TM} = H_\text{obs}R\cdot\bigl(\text{norm. exposure map}\bigr)
\end{align*}
PIB corrected image are obtained by subtracting the PIB map of each TM from the respective filtered photon image. The complete image for TM0 is obtain by co-adding all PIB corrected images. Furthermore, individual background maps are also co-added to form a complete PIB map, which can be found in Appendix \ref{sec:appendix_a_pib_map}. Figure \ref{fig:comparison_of_spf_and_filtered} compares the TM0 filtered image before and after PIB correction.
%
\section{Absorption Correction}
As discussed in Section \ref{sec:background}, X-ray absorption by the ISM must be considered. The methodology outlined in \cite{Willingale2013} is followed. Additionally, scripts for the absorption correction were provided by Angie Veronica (\cite{veronica2020}). At higher energies (\(\gtrsim \SI{0.2}{\kilo\electronvolt}\)), as is relevant for this analysis, absorption from metals play a significant role. 
Assuming solar metallicity, the hydrogen column density \(N_\text{H}\) can used to trace the absorbing material. A cutout of the HI4PI all-sky survey (\cite{HI4PI2016}) is reprojected onto relevant sky tiles to create a neutral atomic hydrogen map (\(N_{\text{HI}}\)map) . Additionally, a molecular hydrogen map (\(N_{\text{H}_2}\)-map) is constructed by dividing the full sky image into \(52 \times 52\) pixel cells, querying \(N_{\text{H}_2}\) values from the Swift homepage\footnote{https://www.swift.ac.uk/analysis/nhtot/index.php (Last accessed: 25.07.2024)}, and distributing these values across each cell. The total \(N_{\text{Htot}}\)-map, constructed by \(N_{\text{Htot}} = N_{\text{H}} + 2N_{\text{H}_2}\), is shown in Appendix B, with \(N_{\text{H}}\) ranging from \(N_\text{Htot, min} = \SI{2.91e+20}{{\centi\meter}^{-2}}\) to \(N_\text{Htot, max} = \SI{1.61e+21}{{\centi\meter}^{-2}}\).
Next, for a each individual \(N_{\text{H}}\)-value in the \(N_{\text{Htot}}\)-map, the expected soft band count rates for TM1 and TM5 are simulated for the model
\begin{align*}
    \text{apec}_{\text{LHB}} + \text{ph}_\text{abs.}\cdot(\text{apec}_{\text{MWH}} + \text{pow}).
\end{align*}
using XSPEC\footnote{https://heasarc.gsfc.nasa.gov/xanadu/xspec/ (Last accessed: 25.07.2024)} with the \texttt{fakeit} command and their respective area (ARF) and response files (RMF). Here, \(\text{apec}_{\text{LHB}}\) represents unabsorbed Local Hot Bubble emission, \(\text{ph}_\text{abs.}\) the absorption along the line of sight, \(\text{apec}_{\text{MWH}}\) the absorbed Milky Way Halo emission, and \(\text{pow}\) the absorbed emission from unresolved point sources (e.g., AGNs). 

A correction factor \(A_{\text{corr}}\) is determined for each value of \(N_\text{H}\) by dividing the simulated count rate for the \(N_{\text{H}}^{\text{tot}}\)-map median \(\bigl(\overline{N_{\text{Htot}}}\bigr)\) by the simulated count rate of the \(N_{\text{H}}\) of interest, hence
\begin{align*}
    A_{\text{corr}}(N_\text{H}) = \frac{\text{simulated count rate}\bigl(\overline{N_{\text{Htot}}}\bigr)}{\text{simulated count rate}\bigl(N_\text{H}\bigr)}.
\end{align*}
Finally, each \(N_\text{H}\) in the \(N_\text{Htot}\)-map is replaced by the correspoding correction factor \(A_\text{corr}\) to create an absorption correction map. Absorption correction maps are created for both TM1 and TM5, which serve as proxies for TM8 and TM9, respectively, due to their similar response files, and will be utilized in the subsequent steps.
\section{Exposure Correction}
The counts image of any given X-ray observation inherently depends on a detector's effective area and the telescope's pointing motion throughout the observation. To obtain meaningfull flux units (e.g. \(\text{cts}\cdot\text{arcsecond}^{-2}\text{s}^{-1}\)), these effects must be considered. This is achieved by dividing the count image by an exposure map, thereby rescalling all segments of the count image to the same relative exposure \cite{davis2001formal}. However, an exposure map must be created separately for TM8 and TM9 due to three reasons: first, TM9 uses a narrow energy band, which lowers its expected count rate; second, this narrower energy band necessitates a different absorption correction map; third, TM8 and TM9 have very distinct response files because of their different filter configurations. Thus, both exposure maps cannot simply be combined but must be corrected for these effects. The procedure outlined in \cite{Reiprich2021} will be followed. First, the exposure map of TM8 and TM9 are divided by their respective absorption correction maps to obtain an absorption-corrected exposure map (\(\text{exmap}_\text{TM8, corr}\), \(\text{exmap}_\text{TM9, corr}\)). Second, the ration of PIB-corrected count rates of TM8 and TM9 is used to define a correction factor 
\begin{align*}
    E_\text{corr} = \frac{\text{PIB corr. count rate(TM9)}}{\text{PIB corr. count rate(TM9)}}
\end{align*}
The total absorption-corrected exposure map for TM0 is then given by
\begin{align*}
    \text{exmap}_\text{TM0, corr} = \text{exmap}_\text{TM8, corr} + E_\text{corr}\cdot\text{exmap}_\text{TM9, corr} 
\end{align*}
Dividing the PIB-corrected TM0 image by the complete absorption-corrected exposure map leads to the final filtered, PIB-corrected, absorption-corrected, exposure-corrected image.
\section{Wavelet filtering and Point Source Removal}
It is necessary to remove the emission from point-like sources (e.g. AGNs) to prevent interference with the cluster emission under study. This is achieved using the wavelet filtering pipeline as described in \cite{Pacaud2006}. Wavelet transformation decomposes an image \(I(x, y)\) into coefficients \((w_1, \ldots, w_n, c_n)\)
\begin{align*}
I(x, y) = c_n(x, y) + \sum_{j=1}^{n}w_j(x, y).    
\end{align*}
Here, \(c_n(x, y)\) is the smoothed image, and \(w_j(x, y)\) represents the contribution of a wavelet function at a scale \(j\) and position \((x, y)\) to the total image. By retaining only the coefficients that satisfy
\begin{align*}
|w_j(x, y)| > k\sigma_j,
\end{align*}
where \(\sigma_j\) is the standard deviation at scale \(j\) and \(k\) is a clipping factor, and then applying the inverse wavelet transformation, an image is obtained that includes only significant scales, i.e. those not due to noise (\cite{Stark1998}). The resultant image is referred to as a wavelet-filtered image. In the pipeline being used, the absorption-corrected exposure map, \(\text{exmap}_\text{TM0, corr}\), and the raw photon image are used to account for previous data reduction and to statistically handle the Poisson noise. After wavelet filtering, a source catalogue is obtained running \texttt{SExtractor} (\cite{Bertin1996}). This is enabled by the significant noise reduction and smoothed background achieved by wavelet filtering. The extended source around the center of NGC1550 is manually removed from the catalogd. A cheese mask is then created from this source catalog and applied to \(\text{exmap}_\text{TM0, corr}\). The unrelated emission around IC366 is also manually added it to the cheese-mask. Rerunning the wavelet filtering with the cheese-masked exposure map reduces ringing artifacts in the wavelet-filtered image, which are typically caused near discontinuities (steep flux gradients due to bright point-like sources). Figure \ref{fig:comparison_wvl_filtered}compares the wavelet filtering before and after application of the cheese-mask to the exposure map.
%
\begin{figure}[htbp]
    \centering
    \begin{subfigure}[b]{0.48\textwidth}
        \centering
        \includegraphics[width=\textwidth]{data_reduction/wlt_filtered.pdf}
        \caption{Before cheese-mask}
        \label{fig:wlt_filtered}
    \end{subfigure}
    \hfill
    \begin{subfigure}[b]{0.48\textwidth}
        \centering
        \includegraphics[width=\textwidth]{data_reduction/wlt_filtered_cheesed.pdf}
        \caption{After cheese-mask}
        \label{fig:wvl_filtered_cheesed}
    \end{subfigure}
    \caption{Corrected wavelet filtered image before (left) and after (right) applying the cheesemask created with \texttt{SExtractor}. The left image is normalized such that the ringing artifacts are most prominent. The cheese-masked imaged has significantly reduced ringing artifacts.}
    \label{fig:comparison_wvl_filtered}
\end{figure}





.
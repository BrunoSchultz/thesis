% !TEX root = mythesis.tex

%==============================================================================
\chapter{Data Reduction}
\label{sec:data_reduction}
%==============================================================================
In the following section, the underlying data shall be reduced and corrected for the various effects and contamination sources explained in Section \ref{sec:background}. Data from eRASSX is utilized for for all TMs (1-7) using \textit{eROSITA} pipeline processing version c010. The galaxy group NGC1550 is located in skytile 065087. In addition, the surrounding skytiles 062084, 062087, 062090, 065084, 065090, 068084, 068087, 068090 are used to encompass regions up to \(\sim 3R_{200}\). The data reduction is performed with the software HEASoft version XXX and the extended Science Analysis Software System (eSASS 4DR1). Images were created using astropy.
\section{Data preparation and filtering}
For each skytile, the \textit{eSASS} task \texttt{evtool} is employed with \texttt{pattern=15} to select single, double, triple, and quadruple event patterns, and \texttt{flag=0xc00fff30} to remove bad pixels and CCD corners. Subsequently, soft proton flares are identified and mitigated through the following process: using the \texttt{flaregti} task, light curves are generated with \SI{20}{\second} time bins within the energy range of \SIrange{5}{10}{\kilo\electronvolt}. A \(3\sigma\) threshold is determined; time intervals exceeding this threshold indicate periods of elevated count rates likely caused by soft proton flares. The task \texttt{flaregti} is then rerun using this threshold to establish good-time-intervals (GTIs) excluding these flare periods, which are then applied using \texttt{evtool} with the \texttt{gti="FLAREGTI"} parameter. 
\section{Image Creation}
To detect emission structures, one usually focuses on soft the X-ray band. In this analysis, we use the \SIrange{0.2}{2.3}{\kilo\electronvolt} energy band. Due to the light-leak in TM9 , however, the energy range is restrictated to \SIrange{0.8}{2.3}{\kilo\electronvolt}, while for TM8 it remains \SIrange{0.2}{2.3}{\kilo\electronvolt}.












% !TEX root = mythesis.tex

%==============================================================================
\chapter{Conclusion}
\label{sec:conclusion}
%============================================================
\section{Results and discussion}
In the surface brightness analysis in Chapter \ref{sec:data_analysis}, a contour plot of the corrected, cheesemasked wavelet filtered image in the soft-band with an overlaid galaxy distribution was provided. From this, it was established that the galaxy group NGC1550 appears relaxed and spherically symmetrical. However, within \(810''\), some indications of asymmetry were visually identified using a contour plot of the corrected image. Emission was found to drop off to background levels at \(R_{500}\), with a somewhat complex background noted, which can be attributed to the Orion-Eridanus superbubble. No obvious correlation between the X-ray contours and the galaxy distribution could be ascertained.

To more accurately characterize the morphology of NGC1550 in the X-ray, a surface brightness analysis was performed. A \(\beta\)-model was employed to characterize the full azimuthal profile, as shown in Figure \ref{fig:tot_azimuthal_beta_model}. The model fitting yielded a \(\beta\)-value of \(0.478 \pm 0.008\) and a core radius of approximately \(r_c = 60'' \sim (15\pm2)\,\text{kpc}\), with a reduced chi-squared value of \(0.96\), indicating a good fit. This result was consistent with the findings of various authors, showing that the eROSITA view of NGC1550 aligns with previous findings. However, the analysis also highlighted a slight discrepancy between the observationally estimated background and the fitted background level, suggesting complexities in background estimation that could affect the surface brightness measurements.

The residual image indicated that the beta model underestimates the core emission, as is common, but no other features were detected. Attempts to use a two beta model were unsuccessful, likely due to the large width of the annuli and the small inner core component, which, from previous studies (\cite{Kawaharada_2009}), is of order of eROSITA's angular resolution. A successful two-\(\beta\) model might have been possible by, for example, fixing the inner core radii or convolving the model with the eROSITA point spread function, but this was attempted within the scope of this thesis.

Moreover, the sectorial surface brightness analysis across the northern, southern, eastern, and western sectors, as well as the combined northern-southern and eastern-western sectors, revealed no significant deviations from azimuthal symmetry. Individual beta fits had slightly lower quality (\(\chi^2 \lesssim 0.8\)), likely due to the low count numbers per annulus. Although regional surface brightness profiles did not deviate significantly from the azimuthal profile, their fit parameters, particularly the \(\beta\) parameters and core radii across sectors, varied significantly. It was speculated that this could be due to a poor choice of the SB-center, creating a significant discrepancy between the north and south emission for the first data point at \(\sim 60''\). 

A potential solution would be a two-dimensional fit of the full azimuthal beta model, allowing free SB-center fitting and better estimation, or simply a more careful selection of the SB-center. Ignoring this issue, however, deviations in fit parameters might be taken as an indication, that galaxy groups do not exhibit perfect azimuthal symmetry, with infalling matter causing inevitable deviations. However, due to the complex background obscuring diffuse group emission in the outskirts and the aforementioned fitting issues, it is challenging to definitively conclude if the findings of this thesis support this view.
\section{Outlook}
Given that only eRASS:1 data was utilized, several features that were qualitatively observed, such as a slight emission dip in the western sector between \(\sim 390''\) and \(\sim 810''\), could not be quantitatively verified. With additional data, such as from eRASS:3, these features could be more accurately interpreted and their significance better assessed. The increased data would also allow for finer annuli in the surface brightness analysis due to the higher count rates in each annulus bin, facilitating better feature identification and potentially simplifying a two-beta model fit.

Furthermore, future work could leverage eROSITA to perform detailed spectroscopy of the outskirts of NGC1550. X-ray spectroscopic measurements can provide insights into the enrichment of the intracluster medium (ICM) and the historical processes of galaxy groups or clusters \cite{Liu2020}. Understanding the distinctions in enrichment processes between galaxy groups and clusters remains an ongoing challenge. eROSITA's wide field of view presents a unique opportunity to study these processes in greater detail, enhancing our knowledge of the chemical composition and evolutionary history of NGC 1550 and similar systems. (TODO: more precise?)
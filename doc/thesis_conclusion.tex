% !TEX root = mythesis.tex

%==============================================================================
\chapter{Conclusion}
\label{sec:conclusion}
%============================================================

In the surface brightness analysis in Chapter \ref{sec:data_analysis}, a \(\beta\)-model was employed to characterize the full azimuthal profile, as shown in Figure \ref{fig:tot_azimuthal_beta_model}. The model fitting yielded a \(\beta\)-value of \(0.478 \pm 0.008\) and a core radius of approximately \(r_c = 60''\) (\(\sim (15\pm2\))), with a reduced chi-squared value (\(\chi^2 / \text{d.o.f}\)) of 0.96, indicating a good fit. This results was found to be consistent with the findings of various authors,   


However, the analysis also highlighted a slight discrepancy between the observationally estimated background and the fitted background level, suggesting complexities in background estimation that could affect the surface brightness measurements.

A notable observation from the sectorial surface brightness analysis is the variation in the \(\beta\)-parameters and core radii between different sectors (north, south, east, and west). The east-west sectors exhibit a sharper emission drop-off compared to the north-south sectors, aligning with the visual elongation noted in the core region. The deviations in the \(\beta\)-parameters and core radii across different sectors highlight the impact of local astrophysical phenomena on the observed symmetry and radial profiles. These results align with previous studies, confirming that galaxy groups, while generally symmetrical, exhibit inherent structural variations.

The residual image analysis further supports these findings. The residuals, derived by subtracting the scaled \(\beta\)-model from the corrected image, indicate that the \(\beta\)-model slightly underestimates the emission around the core. This suggests that a single \(\beta\)-model did not fully capture the core's emission. necessitating consideration of multi-component models or more sophisticated modeling techniques to better describe the inner regions.

Despite the overall consistency of the surface brightness profiles within the \(R_{500}\) radius, the analysis reveals significant insights into the structural complexities of the galaxy group. 

Furthermore, the background estimation process underscores the challenges associated with accurately determining background levels in astrophysical observations. The observed gradient in background emission, with a lower background in the north compared to the south, emphasizes the need for meticulous background subtraction techniques to minimize their impact on the surface brightness analysis. The consistency of the total background level with the northern and southern regions within a \(1\sigma\)-interval provides confidence in the robustness of the background estimation approach, despite the noted complexities.

Future work could focus on utilizing better modeling techniques to describe the surface brightness profile. A successful two-beta-model fit can be possible by for example fixing the core radii, utilizing more data from the inner region and utilizing more refined techniques such as convolving the beta model with the eROSITA point spread function. 


% !TEX root = mythesis.tex

%==============================================================================
\chapter{Results and discussion}
\label{sec:conclusion}
%============================================================
In the surface brightness analysis in Chapter \ref{sec:data_analysis}, contour plots of the corrected, cheesemasked wavelet filtered image with an overlayed galaxy distribution were provided. From this, it was established that the galaxy group NGC1550, at a first glance, appears to be relaxed and spherically symmetrical, though within \(810''\) some indications of assymetry were visually pointed out using a contour plot of the corrected image but were not very prominent.  Emission was found to drop off to background levels at \(R_{500}\) and a somewhat complex background was noted. No obvious correlation between the X-ray contours and the galaxy distribution could be ascertained. 

To more accurately characterize the morphology of NGC1550 in the X-ray, surface brightness analysis was perfomed. A \(\beta\)-model was employed to characterize the full azimuthal profile, as shown in Figure \ref{fig:tot_azimuthal_beta_model}. The model fitting yielded a \(\beta\)-value of \(0.478 \pm 0.008\) and a core radius of approximately \(r_c = 60''\) (\(\sim (15\pm2\))), with a reduced chi-squared value (\(\chi^2 / \text{d.o.f}\)) of 0.96, indicating a good fit. This result was found to be consistent with the findings of various authors, indicating that the eROSITA view of NGC1550 is consinstant with previous findings.
However, the analysis also highlighted a slight discrepancy between the observationally estimated background and the fitted background level, suggesting complexities in background estimation that could affect the surface brightness measurements.

The residual image indicated that the \(\beta\)-model slightly underestimates the core emission, a common issue, but no other features were detected. Attempts to use a two-\(\beta\) model were unsuccessful, probably due to the large width of the annuli and the small inner component's order of eROSITA's angular resolution. A successful two-\(\beta\) model might be possible by fixing core radii or convolving the model with the eROSITA point spread function but, due to time constraint, this could not be attempted within the scope of this thesis.

Moreover, the sectorial surface brightness analysis showed that along the northern, southern, eastern and western and combined northern, southern and eastern,western sectors no obvious deviations from azimuthal symmetry could be noted. The individual beta fits were slightly lower in quality, with \(\chi^2 \lesssim 0.8\), but this was pointed out to be likely due to low amount of counts within each anulli. Though the surface brightness profiles of each region did not differ significantly from the azimuthal profile, the individual fit parameters and their respective \(\beta\) differed somewhat significantly from each other, specifically the variation in the \(\beta\)-parameters and core radii between different sectors. It was noted that these discrepancy could be due to biassing the surface brightness to a certain region. The estimated SB-center of \(\SI{64.909}{\degree}\) and \(\SI{2.414}{\degree}\) seems to be too far north. This problem is made worde by the fairly wide annuli of \(1'\) meaning that excess emission of the first data point might be significantly affecting the fits.  Possibility of fixing would be to attempt at two-dimensional for the full azimuthal beta model fit, which would allow the SB-center to be fitted freely and better estimated. 

Future work could utilize eROSITA to perform spectroscopy of the outskirts of NGC1550. Spectroscopic measurement in the X-rays opens up the study of the enrichment of the ICM and thus the group or cluster history (cite{Liu2020}). It is widely accepted that metals in ICM of galaxy groups and clusters are primarily supplied by member galaxies over cosmic time, with minimal diffusion from their source.  Previous studies (\cite{Kawaharada_2009}) have examined the metal abundance distribution in NGC1550, suggesting that the observed properties can be explained by a galaxy merger scenario, where the central galaxy (NGC1550) has accumulated metals from earlier merging events. This scenario accounts for both the current optical luminosity and the central decrease in the integrated mass-to-light ratio (IMLR) profile that they found. Further investigations with eROSITA could refine our understanding of these metal distribution profiles and the ICM enrichment processes, also in comparison to fossil-like groups such as NGC1550. Additionally, the distinctions in enrichment processes between galaxy groups and clusters are still not well-defined. I has been previously suggested proposed that Type Ia supernovae might have a more significant role in enriching groups compared to clusters. eROSITA's wide field of view offers a unique opportunity to explore these processes and enhance our insights into the characteristics of NGC1550 and similar systems.










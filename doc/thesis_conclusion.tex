% !TEX root = mythesis.tex

%==============================================================================
\chapter{Conclusion}
\label{sec:conclusion}
%============================================================
The aim of this thesis was the characterization of the X-ray morphology of the galaxy group NGC1550 through detailed surface brightness analysis. After correcting and reducing the raw photon data obtained from eRASS:1 in Chapter \ref{sec:data_reduction}, Chapter \ref{sec:data_analysis} began with qualitative inspection of the corrected data. A contour plot of the corrected, cheese-masked wavelet filtered image in the soft-band with an overlaid galaxy distribution was shown (Figure \ref{fig:contour_wvl_filtered}). From this figure, it was visually established that the galaxy group NGC1550 appears relaxed and spherically symmetrical. Furthermore, using a contour plot of the corrected soft-band image (Figure \ref{fig:contour_fully_corrected}), some indications of asymmetry were visually identified within \(810''\). Emission was found to drop off to background levels at \(R_{500}\) and a complex background was noted, which can be attributed to the Orion-Eridanus superbubble. No obvious correlation between the X-ray contours and the galaxy distribution could be ascertained.

To more accurately characterize the morphology of NGC1550 in the X-ray, surface brightness analysis was performed. A beta model was fitted to characterize the full azimuthal profile, as shown in Figure \ref{fig:tot_azimuthal_beta_model}. The model fitting yielded a \(\beta\)-value of \(0.478 \pm 0.008\) and a core radius of \(r_c = (60 \pm 5)'' \approx (15\pm2)\,\text{kpc}\) with a reduced chi-squared value of \(0.96\), indicating a good fit. This result was consistent with the findings of previous studies, showing that the eROSITA view of NGC1550 aligns with previous findings. The full azimuthal surface brightness profile showed no significant deviations from spherical symmetry. Moreover, the sectoral surface brightness analysis across the north, south, east, and west sectors, as well as the combined north-south and east-west sectors, also revealed no significant deviations from azimuthal symmetry beyond \(390''\).

The residual image indicated that the beta model does not described emission within \(R_{500}\) accurately, underestimating the core emission, as is common, and no features were detected. Attempts to use a two beta model were unsuccessful, likely due to the large width of the annuli and the small inner core component, which, from previous studies (\cite{Kawaharada_2009}), is of order of eROSITA's angular resolution. 

Within \(390''\), the qualitatively observed east-west elongation could not be quantitatively confirmed by the surface brightness analysis. Although the sectoral surface brightness profiles did not differ significantly from the azimuthal profile, their fit parameters often varied significantly and the individual beta fits of each sector were of slightly lower quality (\(\chi^2 \lesssim 0.8\)). It was speculated that these issues were caused by the fairly wide \(1'\) anulli and due to a poor choice of the SB-center, resulting in a significant discrepancy between the north and south emission for the first data point at \(\sim 1'\). The surface brightness analysis also highlighted a slight discrepancy between the observationally estimated backgrounds and the fitted background levels, suggesting that background estimation could have been performed with greater care.   
\section{Outlook}
After analyzing the results of this thesis, some suggestions for improvement can be made. The SB-center could have been determined more precisely, either manually or, better, by a two-dimensional beta fit of the full azimuthal profile, allowing it to be freely fitted. A comparison using the brightest pixel as the SB-center is ongoing work and will be presented at a later date. Better background estimation using different background regions could also have been attempted, and visualization of the background gradient by plotting the background emission of each circular region may also be interesting. 

Furthermore, because only eRASS:1 data was utilized, several features observed qualitatively, such as a slight emission dip in the western sector between \(\sim 390''\) and \(\sim 810''\), could not be verified quantitatively. With additional data, such as from eRASS:4, these features could be more accurately interpreted and their significance better assessed. The increased data would also allow for finer annuli in the surface brightness analysis due to the higher count rates in each annulus bin, thus facilitating feature identifications. Although no evidence was found in this work, additional data with better statistics may reveal other features in the soft-band, such as filaments in the outskirts \citep{Cen_1999}. Additionally, enabled by eROSITA's wide field of view and great sensitivity in the soft-band, future work could study superbubbles such as Orion-Eridanus, as has been done in previous studies with ROSAT data \citep{Krause_2014}. 
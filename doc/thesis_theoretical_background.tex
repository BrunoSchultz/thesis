% !TEX root = mythesis.tex

%==============================================================================
\chapter{Theoretical Background}
\label{sec:theoretical_background}
%==============================================================================
\section{Clusters and groups of galaxies}\label{sec:clusters}
Throughout the Universe, galaxies are not distributed homogeneously but are instead aggregated into massive cosmic structures known as galaxy groups or galaxy clusters. Galaxy clusters -- the largest relaxed structures in the Universe -- typically have masses exceeding \(M \gtrsim 3 \times 10^{14} M_{\odot}\), whereas galaxy groups have masses around \(M \sim 3 \times 10^{13} M_{\odot}\) (\cite{Schneider_2006}). Advancements in X-ray astronomy have demonstrated that these structures are significant sources of X-ray radiation (\cite{Cavaliere_1971}). This emission is well understood to originate from a hot intergalactic gas known as the intracluster medium (ICM), which is characterized by temperatures in the range of \SIrange{e7}{e8}{\kelvin} and constitutes the primary baryonic component of galaxy clusters (\cite{Schneider_2006}).
%
\subsection{The Intracluster Medium (ICM)}
Within the deep gravitational wells of galaxy clusters, the temperatures become sufficiently high to fully ionize lighter elements and partially ionize heavier elements, resulting in the formation of a plasma. This hot, diffuse, and optically thin plasma, known as the Intracluster Medium (ICM), emits significant amounts of X-ray radiation. X-ray analysis of the ICM have enabled a wide variety of cosmological studies, including large-scale structure formation in the Universe (\cite{KravtsovBorgani2012}).
%
\subsection{Emission Processes within the ICM} 
A key principle of electrodynamics is that accelerated charges radiate energy. This radiation is referred to as bremsstrahlung or "free-free" when a free charged particle, typically an electron, is accelerated by the electric field of other charges, usually ions. In the ICM, this process predominates at temperatures above \(k_B T_\text{e} \gtrsim \SI{2}{\kilo\electronvolt}\), where the total emissivity at solar metallicity scales approximately as 
\begin{align*}
    \epsilon_{\text{ff}} \propto T_\text{e}^{\frac{1}{2}} n_\text{e},
\end{align*}
with \(n_\text{e}\) and \(T_\text{e}\) as the electron number density and temperature, respectively. At lower temperatures (\(k_B T \lesssim \SI{2}{\kilo\electronvolt}\)), line emission becomes significant, with the emissivity being roughly described by 
\begin{align*}
    \epsilon \propto T_\text{e}^{-0.6} n_\text{e}.
\end{align*}
%
\subsection{The galaxy group NGC1550}
\textcolor{red}{Insert cool stuff about cluster here}
%
\section{eROSITA}
The extended ROentgen Survey with an Imaging Telescope Array (eROSITA) is a highly sensitive, wide-field X-ray telescope designed to capture deep and precise images across large areas of the sky. Mounted on the Spektrum-Roentgen-Gamma (SRG) observatory in a halo orbit around the second Lagrange Point, eROSITA operates within the 0.2 to 10.0 keV energy range. It is the first instrument to perform an all-sky imaging survey in the hard X-ray band (2.0 to 10.0 keV). In the soft X-ray band (0.5 to 2.0 keV), eROSITA boasts a sensitivity that is approximately 20 times greater than that of its predecessor, the ROSAT All-Sky Survey. eROSITA features seven identical mirror modules, known as Telescope Modules (TMs), each with 54 mirror shells in Wolter-I geometry and a 1.6-meter focal length. Five TMs (TM1, TM2, TM3, TM4, TM6) have aluminum on-chip optical light filters and are collectively referred to as TM8. The remaining two TMs (TM5, TM7), designed for low-energy spectroscopy, lack these filters and are referred to as TM9. (\cite{Predehl2021}). Collectively, TM8 and TM9 are referred to as TM0.

\todo{How many surverys; which are relevant for my analysis} 
%
\section{Skybackground and contamination sources}\label{sec:background}
For a thorough analysis of X-ray photons, it is essential to carefully consider both external background and internal instrumental contamination effects. The following section will provide a brief overview of the most important factors relevant to this analysis.
\paragraph*{Cosmic X-ray Background (CXB):} The Cosmic X-ray background comprises multiple sources, including diffuse, unabsorbed thermal emissions from the Local Hot Bubble, a plasma cavity surrounding the Sun, and absorbed thermal emissions from the Galactic halo (\cite{galeazzi2006xmm}). Additionally, it includes discrete extragalactic sources, predominantly unresolved AGNs  (\cite{brandt2005deep}). The diffuse component is more prominent in the lower energy band \(\sim\SI{1}{\kilo\electronvolt}\), while the extragalactic sources dominate at higher energies.
\paragraph*{Non-X-ray Background (NXB):} The non-X-ray background consists of two main components: highly variable soft protons flares from the solar corona and Earth's magnetosphere, which can be focused onto detectors, and energetic Galactic Cosmic Ray (GCR) primaries, which interact with the detector to produce secondary particles. While primary GCR events can be mostly discarded by onboard processing, the secondary particles deposit charge in the detector, making it challenging to distinguish them from true X-ray events. (\cite{Bulbul_2020})
\paragraph*{eROSITA light leak:} Shortly after the launch of eROSITA, it was observed that CCDs lacking an on-chip filter (TM9) recorded a notably higher number of events. This was attributed to optical and ultraviolet light from the Sun entering the CCD through an unidentified gap in the detector shielding and was subsequently termed \enquote{light-leak} (\cite{Predehl2021}). 
\paragraph*{\(N_\text{H}\) absorption:} 
As X-rays travel to the detector, they undergo photoelectric absorption in the interstellar and intergalactic medium. The cross-section \(\sigma \propto E^{-3.5}\) is inversely proportional to energy, causing a bias toward harder X-rays, as they interact less. Additionally, \(\sigma \propto Z^5\) making metal abundance crucial for energies \(\gtrsim \SI{0.2}{\kilo\electronvolt}\).

{\renewcommand{\arraystretch}{1.3}
\begin{table}[htbp]
\centering
  \sisetup{table-format=1.2e-1,table-number-alignment=center}
\caption{Fit Parameters}
\begin{tabular}{l S S S S}
\toprule
  Region &         1 &        2 &        3 &         4 \\
\midrule
 West & 2.38(28)e-5 & 5.47(24)e-1 & 1.11(14)e2 &  1.01(6)e-7 \\
 East & 2.31(28)e-5 & 4.96(17)e-1 &  9.2(12)e1 &  1.05(6)e-7 \\
North & 1.69(30)e-5 & 4.53(19)e-1 &  8.1(16)e1 & 8.08(91)e-8 \\
South & 4.72(70)e-5 & 4.99(14)e-1 & 6.67(99)e1 & 9.53(60)e-8 \\
\bottomrule
\end{tabular}
\end{table}
}
% !TEX root = mythesis.tex

%==============================================================================
\chapter{Theoretical Background}
\label{sec:theoretical_background}
%==============================================================================

\section{Clusters and groups of galaxies}
Throughout the Universe galaxies are not homogeneously distributed, but rather are aggregated in massive cosmic structures called galaxy groups or galaxy clusters. Galaxy cluster feature masses typically surpassing \(M \gtrsim  \SI{3e14}{\solarmass}\), while galaxy groups lie closer to \(M \sim \SI{3e13}{\solarmass}\) (\cite{Schneider_2006}). Advancements in X-ray astronomy have revealed that these structures serve as significant emitters of X-ray radiation (\cite{Cavaliere_1971}). It is currently well understood that this emission stems from a hot intergalactic gas known as the intracluster medium (ICM), characterized by temperatures approximately in the \SIrange{e7}{e8}{\kelvin} range. Moreover, it is widely accepted that the ICM is the primary baryonic component of a galaxy cluster, enabling the study of a variety of cosmological (\cite{Kaiser_1986}) and astrophysical processes (\cite{Lovisari&Reiprich_2018}). In particular, X-Ray analysis lead to a deeper understanding of dynamical disturbances, such as mergers, which strongly change the morphology of the ICM. (\cite{Bykov_2015})
%
\subsection{Emission Processes}
% !TEX root = mythesis.tex

%==============================================================================
\chapter{Theoretical Background}
\label{sec:theoretical_background}
%==============================================================================

\section{Clusters and groups of galaxies}
Throughout the Universe galaxies are not homogeneously distributed, but rather are aggregated in massive cosmic structures called galaxy groups or galaxy clusters. Galaxy clusters feature masses typically surpassing \(M \gtrsim  \SI{3e14}{\solarmass}\), while galaxy groups lie closer to \(M \sim \SI{3e13}{\solarmass}\) (\cite{Schneider_2006}). Furthermore, advancements in X-ray astronomy have revealed that these structures serve as significant emitters of X-ray radiation (\cite{Cavaliere_1971}). It is well understood that the emission stems from a hot intergalactic gas known as the intracluster medium (ICM), characterized by temperatures in the \SIrange{e7}{e8}{\kelvin} range and constituting the bulk baryonic component of galaxy clusters (\cite{Schneider_2006}).
%
\subsection{The Intracluster Medium (ICM)}
Within the deep dark matter gravitational potential of galaxy clusters, sufficiently high temperatures are achieved to fully ionize lighter elements and partially ionize heavier elements, forming a plasma. This hot, diffuse and optically thin plasma emits copious amounts of X-ray radiation and is called the Intracluster Medium (ICM). In particular, X-ray analysis of the ICM have enabled a wide variety of cosmological studies regarding large-scale structure formation within the Universe (\cite{KravtsovBorgani2012}).
%
\subsection{Emission Processes within the ICM} 
A key result from electrodynamics is that accelerated charges radiate energy.  We refer to this radiation as bremsstrahlung or \enquote{free-free} when a free charged particle, typically an electron, is accelerated by the electric field of other charges, typically ions. In the ICM, this process dominates at temperatures above \(k_B T_\text{e} \gtrsim \SI{2}{\kilo\electronvolt}\), where the total emissivity at solar metallicity scales approximately as \(\epsilon_{\text{ff}} \propto T_\text{e}^{0.5} n_\text{e}\). At lower temperatures \(k_B T \lesssim \SI{2}{\kilo\electronvolt}\), line emission becomes significant, with the emissivity being roughly described by \(\epsilon \propto T_\text{e}^{-0.6} n_\text{e}\). Consequently, in the relevant energy range of \(0.1-2.4 \text{ keV}\), one can approximately write
\begin{align*}
    \epsilon_{0.1-2.4\text{keV}} \propto n_\text{e}^2
\end{align*}\(.

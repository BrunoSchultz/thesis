% !TEX root = mythesis.tex

%==============================================================================
\chapter{Introduction}
\label{sec:intro}
%==============================================================================
Unlike the larger and more massive galaxy clusters, galaxy groups are smaller systems with typical masses around \(3 \times 10^{13} M_{\odot}\) (\cite{Schneider_2006}). Despite their relative modesty, the study of galaxy groups is fundamental to our understanding of the Universe's large-scale structure (LSS), as they likely contain a significant fraction of the total baryonic mass in the Universe (\cite{Peebles1998}). Galaxy clusters and galaxy groups are characterized by a hot, ionized gas known as the intracluster medium (ICM) or the intragroup medium (IGM), which fills the space between the galaxies. This gas emits copius amounts of X-rays making X-ray observations an essential tool for identifying and studying these structures (\cite{KravtsovBorgani2012}).

The galaxy group NGC1550, first linked to the extended X-ray source RX J0419+0225 through the ROSAT All-Sky Survey (\cite{Bohringer_2000}), has been the subject of extensive X-ray analysis. Observations have been conducted using various instruments, including ASCA (\cite{Kawaharada_2003}), XMM-Newton (\cite{Kawaharada_2009}), Chandra (\cite{Sun_2003}), and Suzaku (\cite{Sato_2010}). In this thesis, data from the \textit{extended ROentgen Survey with an Imaging Telescope Array} (eROSITA) shall be utilized characterize the X-ray surface brightness profile of NGC1550 and compare the results with previous studies.

Following a brief overview of the theoretical background in Chapter \ref{sec:theoretical_background}, Chapter \ref{sec:data_reduction} will focus on reducing and correcting the data for various effects. Chapter \ref{sec:data_analysis} will analyze the surface brightness profile of NGC1550, including detailed assessments and a beta model fitting of the full azimuthal profile. Additionally, the analysis will compare the north, south, east, and west sectors against each other and against the full azimuthal profile. Finally, Chapter \ref{sec:conclusion} will present the thesis conclusions, and Chapter \ref{sec:outlook} will offer suggestions for future improvement of the analysis.